% LaTeX file for resume
% This file uses the resume document class (res.cls)
\documentclass[margin,line]{res} 
%\usepackage{helvetica} % uses helvetica postscript font (download helvetica.sty)
%\usepackage{newcent}   % uses new century schoolbook postscript font
\setlength{\textheight}{10.5in} % increase text height to fit on 1-page 
\setlength{\topmargin}{-0.25in}
\begin{document}

%% \name{Kenneth Daily}
%% \address{8602 Palo Verde Road\\Irvine, CA 92617\\Phone: (949) 300-9572\\E-mail: kdaily@ics.uci.edu}

\name{\LARGE Kenneth Daily}
\address{\textbf{E-mail:} kenny@kennydaily.net \textbf{Web:} www.kennydaily.net}
\address{10301 Grosvenor Place Apt 703, North Bethesda, MD 20852}

\begin{resume}


%% \section{Objective}
%% Apply for a fellowship award to support the completion of current projects, write my dissertation, and graduate from the University of California, Irvine in 2011.

\section{Education}
Ph.D., Computer Science, University of California, Irvine, 2011\\%% (Advisors: Prof. Pierre Baldi and Prof. Suzanne Sandmeyer)\\ 
M.S., Bioinformatics, Indiana University, 2006\\%% (Advisor: Prof. Predrag Radivojac)\\
B.S., Informatics, Indiana University, 2004%% (Advisor: Prof. Dennis Groth)

\section{Research\\Interests}
Data mining, visualization and integration, software and analysis pipeline development, machine learning, applied statistics, computational biology, genomics, high throughput sequencing

\section{Skills} 
\textbf{Languages:} R (Bioconductor), Python (Django, Ruffus, Snakemake), GNU Linux shell, Perl (BioPerl), HTML/CSS (Bootstrap), SQL, XML, Java, C/C++\\
\textbf{Applications:} Git, Galaxy, \LaTeX, Sun Grid Engine/PBS, Circos, MySQL\\
\textbf{Bioinformatics:} DNASeq, RNASeq, aCGH, microarray, TFBS prediction\\
\textbf{Methods:} Familiar with frequentist and Bayesian statistics, machine learning

\section{Current work}
I am applying integrative analysis methods to high throughput genomic and transcriptomic data for Merkel cell carcinoma in order to identify tumorigenic factors and potential prognostic and therapeutic targets.

\section{Past Projects}
%% \begin{itemize}
%% \itemsep -2pt % reduce space between items
%% Analysis of Transcriptional Regulation in Skin Cells Related to Wounding 2011
Developed methods for analysis of genome-wide transcriptional regulation, 2011 \\
Analyzed next gen sequencing data of the yeast Ty3 transposon, 2009\\
%% Transcriptional Regulatory Control of Neural Stem Cell Development, 2011\\
Developed compression methods for next gen sequencing data, 2009 \\
Evaluated similarity metrics for chemical search, 2006\\
Predicted posttranslational protein modifications, 2004-2006
%% \end{itemize}

\section{Publications}
\textbf{K Daily}, A Coxon, JS Williams, CR Lee, DG Coit, KJ Busam, I Brownell. ``Assessment of Cancer Cell Line Representativeness using Microarrays for Merkel Cell Carcinoma.'' Submitted, Mar 2014. \\
X Qi, \textbf{K Daily,} K Nguyen, H Wang, D Mayhew, P Rigor, S Forouzan, M Johnston, RD Mitra, P Baldi and S Sandmeyer. ``Retrotransposon profiling of RNA polymerase III initiation sites.'' Gen. Res., Apr 2012. PMCID: PMC3317150 \\
\textbf{K Daily,} VR Patel, P Rigor, X Xie and P Baldi. ``MotifMap: integrative genome-wide maps of regulatory motif sites for model species.'' BMC Bioinfo., Dec 2011. PMCID: PMC3293935\\
\textbf{K Daily,} P Rigor, S Christley, X Xie, P Baldi. ``Data Structures and Compression Algorithms for High-Throughput Sequencing Technologies.'' BMC Bioinfo., Oct 2010. PMCID: PMC2964686\\
SJ Swamidass, CA Azencott, \textbf{K Daily,} P Baldi. ``A CROC stronger than ROC: measuring, visualizing and optimizing early retrieval.'' Bioinformatics, May 2010. PMCID: PMC2865862.\\
%% \textbf{K Daily,} P Radivojac, AK Dunker. ``Intrinsic disorder and protein modifications: building an SVM predictor for methylation.'' IEEE Symposium CIBCB 2005, San Diego, CA, Nov 2005.

\section{Presentations}
``The UISO cell line is not representative of Merkel cell carcinoma tumors.'' Poster, ISMB. Long Beach, CA, 2013.\\
``MotifMap: Genome-Wide Map of Regulatory Binding Sites.'' NLM Informatics Training Conference. Denver, CO, June 2010.\\
``Analysis of the Ty3 Retrotransposon in S. cerevisiae Using Transposition Assays and HTS Technologies.'' NLM Informatics Training Conference. Portland, OR, June 2009.\\
``Data Structures and Compression Algorithms for High-Throughput Sequencing.'' Poster, ISMB. Stockholm, Sweden, June 2009.

\section{Awards}
Fellows Award for Research Excellence, Center for Cancer Research, 2014\\
Cancer Research Training Award, National Cancer Institute, 2011 - present\\
Graduate Dean’s Dissertation Fellowship, UC Irvine, 2011\\
Biomedical Informatics Training Grant, National Library of Medicine, 2006 - 2010\\
Informatics Barwise Scholar, Indiana University, 2004 - 2006

%% \section{Teaching}
%% Teaching Assistant, University of California, Irvine, 2010 - 2011 \\
%% Associate Instructor, Indiana University School of Informatics, 2004 - 2006 \\
%% Course Instructor, Indiana University School of Informatics, Summer 2005
\end{resume}
\end{document}
