% LaTeX file for resume
% This file uses the resume document class (res.cls)
\documentclass[margin,line]{res} 
%\usepackage{helvetica} % uses helvetica postscript font (download helvetica.sty)
%\usepackage{newcent}   % uses new century schoolbook postscript font
\setlength{\textheight}{9.5in} % increase text height to fit on 1-page 
%% \setlength{\topmargin}{-0.25in}
\begin{document}

\name{\LARGE Kenneth Daily}
\address{\textbf{E-mail:} kenny@kennydaily.net \textbf{Web:} www.kennydaily.net}
\address{10301 Grosvenor Place Apt 703, North Bethesda, MD 20852}

\begin{resume}


\section{\sc Objective}
Work in a collaborative environment building tools and performing computational\\
analyses of high-throughput genomic data to support biological research.

\section{\sc Current\\Position}
Postdoctoral Fellow, Dermatology Branch  (July 2011 - Present)\\
Center for Cancer Research, NCI, NIH, Bethesda, MD USA \\
Primary Investigator: Dr. Isaac Brownell, M.D., Ph.D.

\section{\sc Current\\Work}
I apply integrative analysis methods to high throughput genomic data in order to identify tumorigenic factors and potential prognostic and therapeutic targets in the skin cancer Merkel cell carcinoma.
To support this work I also develop and implement reusable pipelines for the analysis of microarray, RNAseq, DNAseq, and aCGH data.

\section{\sc Education}
Ph.D., Computer Science, University of California, Irvine, 2011\\%% (Advisors: Prof. Pierre Baldi and Prof. Suzanne Sandmeyer)\\ 
M.S., Bioinformatics, Indiana University, 2006\\%% (Advisor: Prof. Predrag Radivojac)\\
B.S., Informatics, Indiana University, 2004%% (Advisor: Prof. Dennis Groth)

\section{\sc Research\\Interests}
Computational biology, genomics, high throughput sequencing, software and analysis pipeline development, data visualization and integration, data mining, machine learning, applied statistics

\section{\sc Skills} 
\textbf{Languages:} R (Bioconductor), Python (Django, Ruffus, Snakemake), GNU Linux shell, Perl (BioPerl), HTML/CSS (Bootstrap), Java, C/C++\\
\textbf{Applications:} Git, Galaxy, \LaTeX, Sun Grid Engine/PBS, Circos, MySQL\\
\textbf{Bioinformatics:} DNASeq, RNASeq, aCGH, microarray, TFBS prediction\\
\textbf{Methods:} Familiar with experimental biological techniques, frequentist and Bayesian statistics, machine learning


\section{\sc Past\\Projects}
%% \begin{itemize}
%% \itemsep -2pt % reduce space between items
%% Analysis of Transcriptional Regulation in Skin Cells Related to Wounding 2011 \\
Developed methods for analysis of genome-wide transcriptional regulation, 2011 \\
Analyzed high-throughput sequencing data of the yeast Ty3 transposon, 2009\\
%% Transcriptional Regulatory Control of Neural Stem Cell Development, 2011\\
Developed compression methods for high-throughput sequencing data, 2009 \\
Evaluated similarity metrics for chemical search, 2006\\
Predicted posttranslational protein modifications, 2004-2006
%% \end{itemize}

\section{\sc Publications}
\textbf{K Daily}, T Alexander, I Brownell. ``Sequencing and characterization of Merkel cell polyomavirus integrations in Merkel cell carcinoma tumors.'' In preparation.\\
\textbf{K Daily}, A Coxon, DG Coit, KJ Busam, I Brownell. ``Genome-wide copy number analysis of Merkel cell carcinoma.'' In preparation.\\
\textbf{K Daily}, A Coxon, I Brownell. ``Identification of somatic changes in Merkel cell carcinoma by whole exome sequencing.'' In preparation.\\
\textbf{K Daily}, A Coxon, JS Williams, CR Lee, DG Coit, KJ Busam, I Brownell. ``Assessment of Cancer Cell Line Representativeness using Microarrays for Merkel Cell Carcinoma.'' Submitted, Mar 2014. \\
AS Hopkin, W Gordon, RH Klein, F Espitia, \textbf{K Daily,} M Zeller, P Baldi, B Andersen. ``GRHL3/GET1 and Trithorax Group Members Collaborate to Activate the Epidermal Progenitor Differentiation Program.'' PLOS Genetics, July 2012. PMCID: PMC3400561\\
X Qi$^*$, \textbf{K Daily$^*$,} K Nguyen, H Wang, D Mayhew, P Rigor, S Forouzan, M Johnston, RD Mitra, P Baldi and S Sandmeyer. ``Retrotransposon profiling of RNA polymerase III initiation sites.'' Gen. Res., Apr 2012. PMCID: PMC331715 \let\thefootnote\relax\footnote{$^*$ Co-first authors} \\
\textbf{K Daily,} VR Patel, P Rigor, X Xie and P Baldi. ``MotifMap: integrative genome-wide maps of regulatory motif sites for model species.'' BMC Bioinfo., Dec 2011. PMCID: PMC3293935\\
\textbf{K Daily,} P Rigor, S Christley, X Xie, P Baldi. ``Data Structures and Compression Algorithms for High-Throughput Sequencing Technologies.'' BMC Bioinfo., Oct 2010. PMCID: PMC2964686\\
SJ Swamidass, CA Azencott, \textbf{K Daily,} P Baldi. ``A CROC stronger than ROC: measuring, visualizing and optimizing early retrieval.'' Bioinformatics, May 2010. PMCID: PMC2865862.\\
\textbf{K Daily,} P Radivojac, AK Dunker. ``Intrinsic disorder and protein modifications: building an SVM predictor for methylation.'' IEEE Symposium CIBCB 2005, San Diego, CA, Nov 2005.

\section{\sc Presentations}
``Oncogenomic analysis of Merkel cell carcinoma.'' Poster, ISMB. Boston, MA, 2014.\\
``The UISO cell line is not representative of Merkel cell carcinoma tumors.'' Poster, ISMB. Long Beach, CA, 2013.\\
``MotifMap: Genome-Wide Map of Regulatory Binding Sites.'' NLM Informatics Training Conference. Denver, CO, June 2010.\\
``Analysis of the Ty3 Retrotransposon in \textit{S. cerevisiae} Using Transposition Assays and HTS Technologies.'' NLM Informatics Training Conference. Portland, OR, June 2009.\\
``Data Structures and Compression Algorithms for High-Throughput Sequencing.'' Poster, ISMB. Stockholm, Sweden, June 2009.

\section{\sc Awards}
Fellows Award for Research Excellence, Center for Cancer Research, 2014\\
Cancer Research Training Award, National Cancer Institute, 2011 - present\\
Graduate Dean’s Dissertation Fellowship, UC Irvine, 2011\\
Biomedical Informatics Training Grant, National Library of Medicine, 2006 - 2010\\
Informatics Barwise Scholar, Indiana University, 2004 - 2006

%% \section{Teaching}
%% Teaching Assistant, University of California, Irvine, 2010 - 2011 \\
%% Associate Instructor, Indiana University School of Informatics, 2004 - 2006 \\
%% Course Instructor, Indiana University School of Informatics, Summer 2005

\section{\sc References} 

\textbf{Dr. Isaac Brownell, M.D., Ph.D.}\\
Investigator, Dermatology Branch\\
Center for Cancer Research\\
National Cancer Institute\\
Building 10, Room 12N246\\
Bethesda, MD 20892-1908\\
E-Mail: isaac.brownell@nih.gov\\

\textbf{Professor Pierre Baldi, Ph.D.}\\
School of Information and Computer Sciences\\
4038 Bren Hall\\
University of California, Irvine\\
Irvine, CA 92697-3435\\
E-mail: pfbaldi@ics.uci.edu\\

\end{resume}
\end{document}
